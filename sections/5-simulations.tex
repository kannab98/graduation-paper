\section{Результаты симуляций в модели канала Hotel Lobby}
Эффективность разработанных алгоритмов была исследована с помощью численного
моделирования. Использовалась реалистичная модель канала на основе трассировки
лучей, описанная в стандарте IEEE 802.11ay  \cite{Maltsev2017}. Сценарий 
<<Hotel Lobby>> рассматривался как базовый сценарий. 
Параметры симуляции представлены в таб. \ref{tab:4.3}. 
По умолчанию максимальное количество переотражений в процедуре
трассировки лучей установлено равным 2.  

Обычный сценарий Hotel Lobby предоставляет канал с лучаом прямой видимости (LOS)
(см. рис. \ref{fig:4.30a}). Для того, чтобы убрать луч прямой видимости (NLOS),
посередине комнаты была установлена дополнительная стена в центре комнаты
(см. рис. \ref{fig:4.30b}).

\begin{figure}[ht]
    \centering
    \begin{subfigure}{0.49\linewidth}
        \centering   
        \includegraphics[width=\linewidth]{example-image-a}
        \caption{}
        \label{fig:4.30a}
    \end{subfigure}
    \begin{subfigure}{0.49\linewidth}
        \centering   
        \includegraphics[width=\linewidth]{example-image-b}
        \caption{}
        \label{fig:4.30b}
    \end{subfigure}
    \caption{Схема расположения БС и пользователя в сценариях (\subref{fig:4.30a}) LOS
    и (\subref{fig:4.30b}) NLOS}
    \label{fig:4.30}
\end{figure}



В работе выполняется  Монте-Карло моделирование, пользователь вбрасывается 
случайным образом в закрашенных синим областях на рис. \ref{fig:4.30}. 
Обе антенные решетки пользователя лежат в горизонтальной плоскости. 
Азимут всех вброшенных пользователей также определялась случайным образом. 
Количество независимых экспериментов составило 10000.

Были рассмотрены 4 различных сценария:
\begin{enumerate}
    \item Статичный LOS канал 
    \item Статичный NLOS канал
    \item Быстро изменяющийся NLOS канал 
    \item Случай низкого ОСШ
\end{enumerate}

 В случае быстро меняющегося канала пользователи вращались в горизонтальной
плоскости с угловой скоростью 100 град/с. Дополнительное описание 
сценария c низким ОСШ приведено в разделе \ref{sec:simulations:lowsnr}.  
Другие параметры моделирования и системы приведены в таблице \ref{tab:4.10}.

\begin{table}
    \centering
    \caption{Параметры системы}
    \label{tab:4.10}
    \begin{tabular}{|l|l|}
        \hline
        \textbf{Параметр} & \textbf{Значение} \\ 
        \hline
        Окружение & IEEE 802.11ay Hotel Lobby \\
        \hline
        Несущая частота & 28 ГГц (FR2) \\ 
        \hline
        Ширина полосы & 50 МГц \\ 
        \hline
        Частота дискретизации & 61.44 МГц \\ 
        \hline
        Размер БПФ & 512 \\ 
        \hline
        Число используемых поднесущих & 384 \\
        \hline
        Число поднесущих в пилотном сигнале & 127 (SS burst)\\
        \hline
        Расстояние между поднесущими & 120 kHz \\ 
        \hline
        Температура шума & 300 K \\
        \hline
        Мощность шума на поднесущую & -114 дБм \\
        \hline
    \end{tabular}
\end{table}

Для оценки точности оцененного AOA, производилось его сравнение  с эффективным азимутальным углом
\eqref{eq:4.2} некоторого геометрического луча из модели. 
В случае однолучевых алгоритмов  это в качестве этого луча выбирался сильный
путь распространения. В представленных результатах разница между оцененным AOA и эффективным азимутом
геометрического луча отмечается как <<G-bias>>.
Для оценки точности многолучевых алгоритмов  выполнялась более сложная процедура. 
Во-первых, сортировался список геометрических лучей в порядке убывания коэффициента передачи. 
Далее удалялись те лучи, которые находились в пределах ширины ДН вокруг самого сильного из них.
После этого следующий в списке лучей рассматривался как самый сильный и процедура повторялась до 
конца всего списка.

Обозначим $\phi_1$ и $\phi_2$ как оценку AOA для основного и запасного лучей соответственно. 
Пусть $\vec \Psi$ -- список геометрических AOA, а $\psi_1$ —
АОА сильнейшего геометрического луча. 
Ошибка определения основного луча $\psi_1$ определялась как наименьшая величина 
между $\abs{\phi_1 - \psi_1}$ и  $\abs{\phi_2 - \psi_1}$. 
Пусть наименьшая ошибка определилась на угле $\phi_1$ .
Ошибка определения запасного луча   $\min(\Psi - \phi_2)$, 
т.е. мы найти ближайший геометрический луч и будем считать его опорным для
резервного луча.  

В качестве показателей эффективности рассматривались
следующие метрики: 

\newcommand\CDF{\text{CDF}}
\begin{itemize}
    \item Функция распределения ошибки (\CDF) оценки AOA (G-bias)
    \item Среднеквадратичная ошибка (СКО)
    \item Среднее значение ошибки ($\CDF = 0.5$)
    \item Значение $\CDF$ по уровню 0.8 и 0.9
\end{itemize}

Предполагается, что алгоритмы работают, если ошибка меньше удвоенной ширины
ДН ($25,2^{\circ}$).  СКО учитывает только эксперименты, в которых 
алгоритмы работают успешно. 
Вероятность того, отказа работы алгоритма также измерялась.  

Однолучевые  разработанные алгоритмы сравнивались с
базовым алгоритмом иерархического поиска (см.  раздел \ref{sec.hSearch}). 
В случае многолучевости базовая линия отсутствует.  

Во всех случаях оценивалось типичное значение ОСГ, которое определялось
как отношение мощности сигнала и шума на одну поднесущую при идеальном
формировании луча.

\subsection{Однолучевые алгоритмы: стат. случай в LOS}
\subsection{Однолучевые алгоритмы: стат. случай в NLOS}
\subsection{Однолучевые алгоритмы: быстро меняющийся канал}
\subsection{Однолучевые алгоритмы: низкое ОСШ}

\subsection{Многолучевые алгоритмы: стат. случай в LOS}
\subsection{Многолучевые алгоритмы: стат. случай в NLOS}
\subsection{Многолучевые алгоритмы: быстро меняющийся канал}
\subsection{Многолучевые алгоритмы: низкое ОСШ}