%! TeX root = ../main.tex
\SafetyRules
При выполнении компьютерного моделирования на персональной
электронно-вычислительной машине (ПЭВМ) соблюдалась техника безопасности в
соответствии с СанПин 2.2.2/2.4.1340-03 \cite{SanPin}.  В помещениях для работы на
компьютерах необходимым условием является наличие естественного и искусственного
освещения.  Естественное освещение реализуется через окна, ориентированные
преимущественно на север и северо-восток. Не допускается размещение мест
пользователей ПЭВМ в цокольных и подвальных помещениях.  Искусственное освещение
должно осуществляться системой общего равномерного освещения. Яркость
светильников в зоне углов излучения от 50 до 90 градусов с вертикалью в
продольной и поперечной плоскостях должна составлять не более 200 кд/м$^2$,
защитный угол светильников должен быть не менее 40 градусов. В случае работы
преимущественно с документами, допускается применение комбинированного
освещения: кроме общего устанавливаются светильники местного освещения, которые
не должны создавать бликов на поверхности экрана и увеличивать его освещенность
более 300 лк.  Площадь одного рабочего места для взрослых пользователей должна
составлять не менее 6 м$^2$, а объем – не менее 20 м$^3$.  Для внутренней отделки
помещений должны использоваться диффузно-отражающие материалы с коэффициентом
отражения от потолка – $0.7–0.8$; для стен – $0.5–0.6$; для пола – $0.3–0.5$.
Поверхность пола в помещениях должна быть ровной, без выбоин, нескользкой,
удобной для очистки и влажной уборки, обладать антистатическими свойствами.
Микроклимат в помещениях, где установлены компьютеры, должен соответствовать
санитарным нормам: температура воздуха в теплый период года должна быть не более
23–25 градусов Цельсия, в холодный – 22–24 градуса Цельсия; относительная
влажность воздуха должна составлять 40–60; скорость движения воздуха – 0.1 м/с.

Для повышения влажности воздуха в помещениях следует применять увлажнители
воздуха, заправляемые ежедневно дистиллированной или прокипяченной питьевой
водой. Помещения перед началом и после окончания работы за компьютером следует
проветривать.  Экран видеомонитора должен находиться от глаз пользователя на
оптимальном расстоянии 600-700 мм, но не ближе 500 мм с учетом размеров
алфавитно-цифровых знаков и символов. При непрерывной работе с компьютером
каждые 1-2 часа делать перерыв на 10-15 минут для отдыха и выполнения комплекса
физкультурно-оздоровительных упражнений.