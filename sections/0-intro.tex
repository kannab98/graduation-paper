%! TeX root = ../main.tex
\Introduction

%Massive MIMO is one of the promising techniques to improve spectral efficiency
%and network performance for reaching its targeted multi-gigabit throughput in 5G
%systems. For 5G New Radio (NR) systems, one of the key differences compared to
%4G systems is the utilization of high frequency millimeter wave (mmWave) bands
%in addition to sub-6GHz bands. To keep the complexity and implementation cost
%low, hybrid analog-digital beam-forming with large-scale antenna array has
%become a common design approach to address the issue of higher propagation loss
%as well as to improve spectral efficiency in mmWave communication in 5G NR. The
%5G NR standard is designed to adapt to different beam-forming architecture and
%deployment scenarios Beamforming techniques used in mmWave communication play an
%extremely important role due to specific propagation conditions and high power
%loss. In mmWave communication issues, those appear due to fast and significant
%channel changes are very crucial and critical for overall system performance.
%Thus the beamforming accuracy is a cornerstone of mm technologies.  In typical
%mmWave systems, realignment of beam pair is performed with the help of beam
%search that sweeps through all possible beam pairs periodically and selects the
%best pair, either through an exhaustive search \cite{Giordani2019}. This beam sweep procedure
%takes significant time and is vulnerable to varying channels (such as UE
%rotation or blockage).  So the main target of this project is to develop fast,
%accurate and robust beam management algorithms for mobile UE under practical
%scenarios which will be effective from a computational complexity point of view
%also. 
%The project consists of several stages which are related to different key
%aspects of beam management. Stage 1 is dedicated to literature survey and
%initial studying of effects via realistic channel modeling. It is performed to
%form a general view and reveal existing solutions.  Stage 2 is dedicated to AOA
%estimation algorithm development. Beam tracking algorithm development is
%considered during Stage 3. Finally, Stage 4 is dedicated to beam management
%under blockage. 

Massive MIMO — один из многообещающих методов повышения спектральной
эффективности и производительности сети для достижения целевой пропускной
способности в несколько гигабит/с для систем пятого поколения.
В системе 5G New Radio (NR) есть одно из главных отличий по сравнению с
системой предыдущего поколения (4G) — это использование высокочастотных диапазонов миллиметровых волн (mmWave) в
в дополнение к диапазонам ниже 6 ГГц. 

Чтобы снизить сложность и стоимость реализации, гибридное аналогово-цифровое формирование луча с крупномасштабной антенной решеткой
стать общим подходом к проектированию для решения проблемы более высоких потерь при распространении, а также для повышения эффективности использования спектра в связи миллиметрового диапазона в 5G NR.

Стандарт 5G NR предназначен для адаптироваться к различным архитектурам
формирования луча и различным сценариям поведения.

Методы формирования ДН,используемые в миллиметровом канале, играют чрезвычайно
важную роль из-за специфичных условий распространения и больших потерь
мощности.  Кроме того быстрые изменения канала оказывают сильнейшее влияние на
производительность
такой системы.
Таким образом, точность формирования ДН играет крайне важную роль во всей
технологии мм-связи. 

В типичных системах миллиметрового диапазона поиск пары лучей пользователя (UE) и
базовой станции (BS)
выполняется с помощью периодического перебора пар возможных лучей с выбором
наилучшей пары. Зачастую это делается путем полного перебора \textit{всех}
возможных пар.  Такая процедура требует значительного времени и уязвима для
некоторых каналов, например, вращение пользователя или блокировка пользователя
препятствием. 

Таким образом, основной целью этой работы является разработка быстрого,
точного, надежного, а также эффективного с точки зрения вычислительной сложности алгоритма управления диаграммой направленности
пользователя в практических
сценариях. Поскольку тема достаточно обширная, мы коснемся лишь оценки AOA со
стороны пользователя.
