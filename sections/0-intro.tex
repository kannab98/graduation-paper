%! TeX root = ../main.tex
\Introduction

Развитие стандарта мобильной связи 5G New Radio (NR), разрабатываемого консорциумом
3GPP (\textit{Англ. - 3rd Generation Partnership Project}), тесно связано с развитием
технологии Интернета Вещей (\textit{Англ. - IoT - Internet of Things}). Высокая скорость,
надежность сети, малая задержка, а также возможность массового подключения
"умных" устройств являются важнейшими параметрами, определяющими
производительность системы в целом.

Одни из последних релизов стандарта 5G NR - релизы 15 и 16 обеспечивают
поддержку несущих частот до 52.6 ГГц. С целью расширить поддержку текущего
частотного диапазона FR2 (\textit{Англ. - Frequency Range 2}) до 52.6 - 71 ГГц с
минимальными вносимыми изменениями в систему \cite{intel193259}
\cite{qlcm193229}, группа RAN проекта 3GPP уже исследовала требования для
диапазона 52.6 - 114.25 ГГц \cite{3gpp.38.807}. Помимо этого, была также
исследована возможность расширения частотного диапазона до миллиметровых
волн 71-114 ГГц. Однако в этом диапазоне появляется такое ограничение, как
нелинейный искажения, вызванные работой усилителя мощности (УМ). Несмотря
на значительное продвижение в технологии разработки и проектирования УМ с
использованием новых материалов, все ещё наблюдаются значительные нелинейные
искажения сигнала при использовании стандартной мощности передатчика
\cite{zhang2021}. В данном диапазоне частот УМ может внести значительные
искажения, кардинально снизив производительность
системы. Это особенно заметно для высокоэффективных модуляций, например,
64-QAM и 256-QAM.

Данным эффектом искажения сигнала можно пренебречь в низких диапазонах
частот, таких как FR1 и частично FR2. Рабочую точку УМ в этих диапазонах
можно выбрать таким образом, что на выходе усилителя будет достигаться
необходимая выходная мощность, и при этом УМ будет работать в линейной
области своей характеристики, что минимизирует вносимые в передаваемый
сигнал искажения.

Проблема искажения сигнала на высоких частотах наиболее актуальна при
рассмотрении использования стандарта связи 5G NR в применении к технологии
Интернета Вещей. В данном случае система имеет огромное количество
небольших и простых передающих устройств, таких как датчики, сенсоры, а
также прочие устройства, используемые для обеспечения тесно
связанного окружения. Подобные элементы часто имеют низкокачественные
передающие и усилительные цепи ввиду необходимости общей низкой стоимости
прибора. Также данные устройства должны быть энергоэффективными и
минимизировать общее потребление электроэнергии для возможности создания
инфраструктуры из большого количества отдельных компонент.
Эти факторы являются ключевыми при выборе метода компенсации вносимых
нелинейных искажений, поскольку вспомогательная обработка на передатчике
может внести дополнительные энергозатраты, нежелательные для дешевого,
энергоэффективного устройства.

Проблема компенсации нелинейных искажений, вносимых УМ, рассматривалась во
многих работах, в том числе для низких диапазонов частот
\cite[]{sharath2015,shabany2008,eda2001,maltsev2021,bhat2016,qi2010,gregorio2007}.
Рассматривались различные подходы, в основе которых лежали предварительное
искажение (\textit{Англ. - Pre-Distortion (PD), Digital Pre-Distortion
(DPD)}) сигнала на передатчике с целью "выпрямления" амплитудной
характеристики усилителя. Однако такие методы требуют дополнительной
сигнальной обработки на передатчике, что негативно влияет на
энергоэффективность устройства. В основе другого метода лежит обработка
сигнала на приемнике, когда принятый сигнал подвергается обработке на
стороне принимающего устройства с целью компенсации искажений, внесенных на
УМ передатчика. В качестве методов компенсации используют обратную
характеристику УМ, статистические подходы для определения усредненного
искажения и его дальнейшей компенсации, последовательный метод Монте-Карло
и другие. Также важно отметить необходимость в определенных случаях знать
на приемнике параметры УМ, который расположен на передатчике.

Также важно отметить, что характеристики УМ в миллиметровом диапазоне
частот 100-200 ГГц значительно отличаются от характеристик усилителей в
диапазоне 30-70 ГГц. На более высоких частотах, характеристики значительно
хуже, это означает необходимость применения определенного метода
компенсации искажения для повышения качества передачи информации.

В данной работе предлагается метод компенсации нелинейных искажений
внесенных усилителем мощности на приемнике. Основа метода компенсации
заключается в использовании обратной характеристики УМ, однако перед этим
сигнал определенным образом обрабатывается. Предлагаемый метод может быть
использован для разных типов сигналов(CP-OFDM, DFT-s-OFDM),
в зависимости от рассматриваемой задачи. Также, в рамках данной работы были
исследованы характеристики современных твердотельных УМ в миллиметровом диапазоне. На
основе проведенного исследования, была создана модель для диапазона
частот 100-200 ГГц, которая в дальнейшем использовалась в математическом
моделировании для проверки работоспособности предлагаемого метода.



% \textbf{Черновой план диплома}

% \textit{Краткое описание. Скорее для себя, чтобы самому не забыть в чем
% основная суть работы, в дипломе этого не будет}

% \begin{itemize} \item 5G расширяется и активно исследует суб-ТГц диапазон
%     \item усилители в этом диапазоне не очень, много искажений
%     \item особенно для маленьких дешевых устройств, где важна
%     энергоэффективность
%     \item нужна обработка на приемнике
%     \item предлагаемый метод обработки
%     \item краткий план-содержание работы
%     \end{itemize}





% Дипломная работа о разработке метода компенсации нелинейных искажение
% сигнала на приемнике, возникших в результате работы усилителя мощности на
% передатчике. Усилитель имеет нелинейную амплитудную характеристику, что и
% является причиной искажения сигнала.

% Учитывая планы расширения стандарта 5G в суб-терагерцовый диапазон,
% нелинейность характеристики УМ в этих частотах только ухудшается,
% значительно снижая эффективность системы в целом. Это является основным
% мотивом разработки метода компенсации вносимых передатчиком искажений на
% приемнике.

% Также, компенсация рассматривается именно на приемнике, поскольку в таком
% случае передатчику не нужно производить дополнительную обработку сигнала,
% что важно для энергопотребления. Такой метод может использоваться в
% системах с большим количеством мало-размерных, простых передатчиков,
% которые часто используются в системе интернета вещей.

% \section{Введение (5G, IoT, sub-THz range, PA nonlinearity)}


% \section{Роль усилителя мощности и влияние нелинейности амплитудной
% характеристики} \subsection{Описание принципа работы усилителя мощности}
% \subsection{Влияние нелинейности и искажение различных сигналов на
% приемнике} \subsubsection{Искажение сигнала SC} \subsubsection{Искажение
% сигнала CP-OFDM} \subsubsection{Искажение сигнала DFT-s-OFDM}
% \subsection{Характеристики усилителей в суб-ТГц диапазоне}
% \subsection{Обзор существующих моделей для описания усилителя мощности}
% \subsubsection{Модель Rapp} \subsubsection{Новая модель для диапазона
% частот 100-200 ГГц}

% \section{Метод компенсации нелинейных искажений на приемнике}
% \subsection{Краткое описание принципа работы симулятора канального
% уровня} \subsection{Подход и описание нового метода компенсации
% нелинейных искажений} \subsubsection{Адаптация алгоритма компенсации в
% зависимости от типа используемого сигнала} \subsubsection{Компенсация с
% использованием обратной характеристики усилителя}

% \section{Результаты расчетов и симуляций} \subsection{Модель усилителя
% мощности в диапазоне 30-70 ГГц} \subsection{Модель усилителя мощности в
% диапазоне 100-200 ГГц}

% \section{Заключение}

% \section{Приложение}

% \newpage
% \Introduction

% \textit{В введении указывается актуальность темы, цели и задачи исследования, предметы и
% методы исследования, теоретическую и практическую значимость, научная новизна,
% положения, выносимые на защиту, краткое описание структуры работы.}


% В последние годы Интернет Вещей (IoT), поддерживаемый технологией 5G,
% стремительно развивался в широком спектре применений, обеспечивая связь
% между объектами в автомобильной индустрии, бытовой электронике,
% транспортном, логистическом и производственном секторе. С ростом
% повсеместной интеграции и использования различных малогабаритных
% датчиков, стоимость производства каждого отдельно взятого элемента
% остается критическим аспектом. Относительно низкая цена отдельных
% элементов является ключом к созданию тесно связанной среды, однако это
% может серьезно повлиять на качество радиочастотных цепей, а также на
% общую производительность системы. С расширением 5G до субтерагерцовых
% диапазонов, нелинейные искажения усилителя мощности (УМ) могут
% значительно ограничить производительность системы даже в
% высококачественных устройствах из-за конструктивных ограничений
% усилителя. Было проведено множество исследований с целью уменьшения
% влияние нелинейности как на стороне передатчика, так и на стороне
% приемника. Многие решения предлагают определенные варианты оценки
% эффектов УМ посредством обратной связи, направленной на принятие решения,
% обучения или даже статистической обработки полученного сигнала. Однако,
% зная функцию нелинейности УМ на приемной стороне, обработка может быть
% упрощена до применения обратной функции к эквивалентному сигналу во
% временной области.

% В этой работе мы предлагаем метод компенсации нелинейных искажений УМ на
% стороне RX, который можно настроить для нескольких форм сигналов, таких
% как CP-OFDM и DFT-S-OFDM. Приведенные результаты моделирования
% демонстрируют улучшение характеристик как для субтерагерцовых моделей УМ,
% так и для моделей в диапазоне 30-70 ГГц со значительно лучшими
% характеристиками.



% \section{Описание проблемы и предшествующие решения}
% \textbf{введение из статьи}

% Последние релизы стандарта 5G New Radio (NR) Rel15 и Rel16 поддерживают
% частоты несущих до 52.6 ГГц. Рассматривая работу на частоте свыше 52.6
% ГГЦ, группа RAN проекта 3GPP уже  исследовала требования для диапазона
% частот 52.6 – 114.25 ГГЦ [1], с основной целью расширить  поддержку
% текущего диапазона NR FR2 до 52.6-71 ГГЦ с минимальными изменениями в
% системе [2][3]. Также были исследованы возможности дальнейшего расширения
% в суб-терагерцовый диапазон 71-114 ГГц. В этом диапазоне, несмотря на
% значительное продвижения в технологии разработки и создания УМ, все еще
% наблюдается сильные нелинейные искажения для стандартной разрешенной
% мощности передатчика [4]. Таким образом, искажения, вносимые усилителем
% мощности, могут значительно ограничить производительность системы,
% особенно для высокоэффективных модуляций, таких как 64- и 256-QAM.

% В низких диапазонах частот 5G NR, FR1 и частично FR2, нелинейными
% эффектами УМ можно пренебречь в большинстве случаев, поскольку рабочая
% точка может быть расположена в линейной области, позволяя минимизировать
% искажения передаваемого сигнала. Рассматриваемая проблема значительно
% более актуальна для дешевых трансиверов с низкокачественными
% усилительными цепями. Следует заметить, что количество таких устройств
% может быть очень большим, поскольку дешевые устройства обычно
% используются для создания инфраструктуры Интернета вещей. Данная проблема
% была рассмотрена в нескольких работах [5]-[11], даже для низкочастотных
% диапазонов.

% Для описания искажения амплитуды и фазы при использовании твердотельных
% усилителей мощности широко используется модель Раппа. Модифицированная
% модель Раппа, приведенная в уравнении (1), также включена в базовые
% модели спецификации 3GPP \cite{spec38807}.

% \begin{equation} a=b \end{equation}

% насыщения. A,B,q – параметры кривой искажения фазы. is the small signal
% gain, p is the smoothness factor and ?Vsat is the saturation
% voltage/A?B?qare phase distortion curve parameters.

% Базовые характеристики стандартных УМ в диапазоне частот 30-70 ГГц были
% использованы для вывода математической модели [15], справедливой в
% соответствующей полосе. Однако основываясь на недавних работах
% [4][16][17], можно отметить что характеристики Суб-ТГц УМ даже в полосе
% частот 100-200 ГГц значительно отличаются. Для оценки производительности
% системы для данного диапазона несущих частот, нами была выведена
% усредненная модель УМ на основ последних исследований. Сравнение моделей
% приведено на Figure 1. Основываясь на усредненных параметрах ???, была
% получена модель УМ для диапазона частот 100-200 ГГц.

% На Figure 2 приведены примеры искажения сигнала в следствии действия
% нелинейности УМ для разных видов сигнала. Можно отметить, что для сигнала
% с одной несущей (single carrier - SC), искажения носят детерминированный
% характер в виде амплитудного искажения, которое можно с относительной
% легкостью компенсировать. Однако для сигнала OFDM, нелинейность УМ
% приводит к интерференции между несущими, что результирует в случайном
% характере искажения, которое намного сложнее компенсировать.

% DFT spread OFDM (DFT-s-OFDM) представляет некую промежуточную ситуацию,
% когда присутствуют как и детерминированный характер искажения, так и
% случайный, позволяя произвести компенсацию нелинейности.

% На данный момент существуют два различных подхода для компенсации
% нелинейности УМ. Первый заключается в предварительном искажении сигнала
% перед УМ на передатчике, придавая сигнал определенные свойства,
% минимизирующие влияние нелинейного искажения от УМ. Существует множество
% вариантов обработки для данного подхода, однако все они имеют слабый
% эффект на общей производительности системы, а предварительное искажение
% имеет низкую эффективность на низких значениях IBO [5][6][7].
% Использование предварительного искажения  на передатчике нежелательно и
% на мало-габаритных устройствах, поскольку в этом случае растет объем
% сигнальной обработки, что увеличивает энергопотребление.

% Второй подход заключается в компенсации нелинейных искажений на
% приемнике. Например, в [8] используется статистическая обработка
% принятого сигнала для определения средней степени искажения, что
% используется для дальнейшей компенсации. Работы [5][6][9][10][11][12][13]
% рассматривают теоретический подход для компенсации на приемнике в очень
% обобщенном случае. Несколько методов компенсации влияния нелинейности УМ
% были предложены для OFDM сигнала [11][12][13], где влияние нелинейности
% представляется постоянным комплексным множителем, а также Гауссовой
% шумовой компонентой. Основной задачей в таком случае является определение
% параметров УМ (могут быть как известны изначально, так и определены с
% помощью пилотных сигналов) для компенсации нелинейного искажения.
% Несколько методов были использованы для сигнала SC с одной несущей
% [5][6][9][10], включая….??..., последовательный метод Монте-Карло, а
% также обратную характеристику УМ. В нескольких случаях [9][11][10],
% значения параметров УМ считаются известными на приемнике, что позволяет
% произвести компенсацию искажения. В случаях, когда параметры УМ
% оцениваются, производительность такая же либо хуже.

% \section{Новая идея}

% Как было показано для сигнала SC, и, не менее важно, для сигнала
% DFT-s-OFDM, искажения в результате влияния нелинейности УМ имеют
% детерминированный характер, в добавок к ICI. Зная нелинейную
% характеристику, становится возможным компенсировать детерминированную
% компоненту искажений, улучшая итоговую демодуляцию. Этого можно добиться
% применяя обработку сигнала на приемнике, эквивалентную обратной
% характеристике УМ.

% Обычно, эта функция не известна на приемнике не только потому что
% значения параметров УМ не известны, но и в следствии различных значений
% мощности передатчика. Значение этой мощности напрямую влияет на рабочую
% точку УМ, а значит и на то, как искажается конечный сигнал.

% В нескольких работах [10][11][12], the decision-directed feedback??
% используется для оценки характеристики усилителя мощности. Также, в [8]
% используется статистическая обработка для корректировки алгоритма
% демодуляции принятого искаженного сигнала.

% Однако, более эффективным подходом является использование знаний о
% параметрах УМ, а также его рабочей точки для соответствующей пост
% обработки.

% \subsection{Предлагаемый метод компенсации на приемнике} Ниже приведен
% метод компенсации нелинейных искажений УМ на стороне приемника для
% сигнала CP-OFDM, основанный на использовании обратной амплитудной
% характеристики УМ (Figure 3). 

% Схема компенсации состоит из базовой обработки на передатчике (1),
% которая может включать MIMO прекодинг и transform прекодинг (в случае
% DFT-s-OFDM сигнала),  а также стандартного для OFDM блока обратного
% преобразования Фурье. Сгенерированный OFDM сигнал подается на одну или
% более передающих цепочек, которые могут включать добавление цикличного
% префикса, перенос сигнала на несущую частоту, и, наконец, сигнал подается
% на усилитель мощности (2), работающий на несущей частоте. Отметим, что
% для корректной работы предлагаемой схемы, сигналы на разных антеннах
% должны иметь одинаковую амплитуду (но могут иметь разную фазу). Это
% ограничивает применение данного метода до передачи 1 ранга, даже если
% используется несколько передающих антенн. После прохождения через канал,
% сигнал попадает на приемную цепь состоящую из одной или нескольких
% приемных антенн для дальнейшей обработки (4), которая может состоять из
% преобразования Фурье further maximum ration combining (MRC) and frequency
% domain equalization???. Такая обработка эффективно нивелирует влияние
% частотно-селективного канала, что позволяет использовать обработанный
% сигнал на блоке компенсации нелинейного искажения (5). Этот блок может
% состоять из операции обратного Фурье преобразования (6) для возвращения
% сигнала во временную область, блока обратной нелинейной функции УМ (7),
% который выполняет компенсацию нелинейного искажения, а также блока
% прямого преобразования Фурье для возвращения сигнала в частотную область.

% \subsection{Предположения и результаты симуляции} Для проверки
% эффективности предлагаемого метода, были произведены симуляции канального
% уровня, в которых предложенный метод сравнивался со случаями идеального
% УМ, а также отсутствии компенсации на приемнике для совпадающей модели
% усилителя. Использовалась модель основанная на параметрах существующих
% усилителей мощности в диапазоне частот 30-70 ГГц [15], а также модель для
% 100-200 ГГц полученная в результате данной работы. Расчеты проводились
% для различных параметров системы, таких как расстояние между поднесущими
% (SCS), типом используемого сигнала, кодирование, модуляция и другие.
% Полный список параметров симуляции приведен в Table 1.

% \subsection{Последовательность применения компенсаций} Предложенный метод
% предполагается использовать до компенсации фазового шума (PN), однако
% другие последовательности были рассмотрены и рассчитаны. Применение
% компенсации нелинейности УМ после компенсации PN показало определенное
% ухудшение производительности, поэтому компенсация нелинейности УМ везде
% производилась до компенсации фазовых шумов в большинстве расчетов.
% Примеры результатов, в которых сравнивается разный порядок компенсации
% фазовых шумов и нелинейности усилителя приведены на Figure 10.

% \subsection{Наблюдения}

% • Для модели УМ в диапазоне 30-70 ГГц (подходящего для диапазона АК2 и
% выше) улучшение наблюдается только для модуляций высокого порядка o Для
% SCS 120 кГц, отрицательные эффекты фазового шума значительны, и при таких
% искажениях результаты компенсации нелинейности УМ незначительны o Для SCS
% 480 и 960 кГц, в которых возможна более эффективная компенсация фазового
% шума, в определенный момент влияние нелинейности УМ становится основным
% ограничивающим фактором. В этом случае компенсация нелинейности УМ может
% улучшить результат на несколько дБ или совсем избавиться от искажений,
% внесенных УМ. • Для модели 100-200 ГГц, влияние нелинейности УМ
% увеличивается, и в большинстве случаев превосходит влияние фазовых шумов
% o Предложенный метод компенсации демонстрирует улучшение результата для
% MCS 22 и выше • Не смотря на возможность различных имплементаций,
% диктуемая логикой последовательность компенсации искажений в порядке,
% обратному их появлению в системе оказывается оптимальным. Таким образом,
% искажения должны быть компенсированы в порядке канал, усилитель мощности,
% фазовые шумы.  

% \section{Заключение}

% Описан и продемонстрирован метод для компенсации нелинейных искажений
% усилителя мощности на стороне приемника. В основе метода лежит
% использование обратной амплитудной характеристики усилителя, метод может
% быть применен для нескольких различных типов сигнала и подстроен для
% рассматриваемой задачи. Компенсация была реализована и протестирована на
% симуляторе канального уровня, соответствующего требованиям стандарта 5G
% NR для оценки влияния на производительность системы. Метод применялся для
% существующей модели УМ в диапазоне частот 30-70 ГГц, а также для новой
% модели 100-200 ГГц, основанной на последних исследованиях. В обоих
% случаях, предлагаемый метод продемонстрировал возможность значительно
% улучшить производительность системы в различных условиях (вид сигнала,
% модуляция и др.). Поскольку метод опирается на компенсацию на стороне
% приемника, этот подход может быть эффективен в системе с большим
% количеством простых, дешевых передатчиков с низким энергопотреблением.
% Это позволит устройству передачи снизить общее энергопотребление,
% поскольку в таком случае отсутствует необходимость в предварительной
% обработке сигнала для компенсации нелинейных эффектов усилителя.

