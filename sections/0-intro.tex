%! TeX root = ../main.tex
\Introduction

Massive MIMO — один из многообещающих методов повышения спектральной
эффективности и производительности сети для достижения целевой пропускной
способности в несколько гигабит/с для систем пятого поколения.  В системах 5G
New Radio (NR) есть одно из главных отличий по сравнению с системой предыдущего
поколения (4G) — это использование высокочастотных диапазонов миллиметровых волн
(mmWave) в дополнение к диапазонам ниже 6 ГГц. 

В системах пятого поколения связи 5G New Radio для повышения спектральной
эффективности и производительности  сети чаще всего применяются многоэлементные 
антенные решетки с аналого-цифровым формированием луча.   

Стандарт 5G NR предназначен для адапаптации к различным способам
формирования диаграммы направленности. Методы формирования ДН,
используемые в миллиметровом канале, играют чрезвычайно
важную роль из-за особенностей распространения и больших потерь мощности.
Кроме того быстрые изменения канала оказывают сильнейшее влияние на
производительность системы.  Поэтому, точность формирования ДН
играет крайне важную роль во всей технологии миллиметровой связи. 

В системах миллиметрового диапазона поиск пары лучей пользователя (UE)
и базовой станции (BS) выполняется с помощью сканирования пар
возможных лучей и выбором оптимальной пары. Зачастую это делается путем полного
перебора \textit{всех} возможных пар.  Такая процедура требует значительного
времени и уязвима для быстро изменяющихся каналов таких как, например,
вращение пользователя или блокировка пользователя препятствием. 


Данная работа носит по большей части прикладной характер и основной её целью
является разработка  быстрого, точного и эффективного с
точки зрения вычислительной сложности алгоритма формирования диаграммы
направленности и оценки угла угловой координаты источника излучения. 

% В главе \ref{sec:review} рассмотрены наиболее популярные алгоритмы оценки угла прихода излучения 
% (Angle of Arrival), а также представлены их преимущества и недостатки. 

% % В главе \ref{sec:theory} речь пойдет об основных  

% В главе \ref{sec:singlepath} дано описание алгоритмов оценки AOA, способных оценить только направление прихода самого сильного сигнала. 
% В главе \ref{sec:multipath} представлены алгоритмы оценки AOA, которые могут оценить помимо главного луча ещё как минимум один запасной луч. 

