%! TeX root = ../main.tex
\Conclusion
В ходе данной работы проводилось исследование различных алгоритмов оценки угла
прихода излучения в системе 5G NR с при наличии у пользователя двух антенных
решеток (АР), расположенных на боковых гранях устройства, что является
конструктивной особенностью многих современных средств связи. 

Был получен новый многолучевой метод оценки угла прихода -- многолучевой
алгоритм иерархического поиска с минимизацией среднеквадратичной ошибки
\textit{hSearchMMSE}.

Кроме того, были рассмотрены и модифицированы некоторые существующие алгоритмы:
\begin{itemize}
    \item Алгоритм моноипульса \textit{Auxiliary Beam} (однолучевой и многолучевой)
    \item Алгоритм бинарного поиска \textit{Adaptive Compressive Sensing} (однолучевой и многолучевой)
\end{itemize}

Эффективность разработанных алгоритмов исследовалась в нескольких сценариях и сравнивалась 
с выбранным базовым алгоритмом иерархического поиска
\begin{enumerate}
    \item Статический канал с лучом прямой видимости (LOS)
    \item Статический канал без луча прямой видимости (NLOS)
    \item Быстро меняющийся NLOS канал 
    \item NLOS канал с низким отношением сигнал/шум 
\end{enumerate}
По результатам Монте-Карло моделирования в реалистичной модели канала
IEEE 802.11ay <<Hotel Lobby>> можно привести следующие заключения: 
\begin{itemize}
    \item В статическом случае все рассмотренные алгоритмы превосходят
    базовый алгоритм. Лучшими являются \AuxBeam{} и \hSearchMMSE{}, 
    поскольку они не имеют ошибки квантования. 
    \item В случае быстро меняющихся каналов и SS-burst в качестве опорного
    сигнала, лучшим решением будет \ACS{} алгоритм, поскольку он
    использует более широкие ДН, что способствует меньшему устареванию
    информации о канале во время продолжительного измерения и позволяет
    проследить правильное направление на первых итерациях алгоритма. 
    \item В случае быстро меняющегося канала CSI-RS оказывается более 
    эффективным, чем SS-burst. Лучшим решением в этом случае будет \AuxBeam{}.
    \item При низком ОСШ, все алгоритмы значительно теряют в эффективности. Лучшим 
    алгоритмом, с точки зрения среднего значения ошибки, является \textit{hSearchMMSE}.
    \item В случае многолучевого статического канала, наибольшую эффективность 
    и вероятность нахождения дополнительных лучей обеспечивает \hSearchMMSE{}.
    \item В случае быстро меняющегося многолучевого канала, предлагается использовать
    \AuxBeam{} или \hSearchMMSE{} с пилотными сигналами CSI-RS 
    \item В многолучевом канале с низким ОСШ лучшим решением будет \hSearchMMSE{}
\end{itemize}

