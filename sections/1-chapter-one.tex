%! TeX root = ../main.tex
\section{Обзор методов оценки угла прихода}
\label{sec:review}

\subsection{Характеристики канала связи миллиметровых длин волн}
Миллиметровая модель канала имеет ряд особенностей, которые описаны во множестве
литературных источников и мировых стандартах \cite{Maltsev2010, Maltsev2017,
    Xu2002, Akdeniz2014, Rappaport2015}. Основные особенности следующие:
\begin{itemize}
    \item Малое влияние дифракции
    \item Высокие потери в канале связи
    \item Потери на шероховатостях отражающих поверхностей
    \item Пути распространения могут быть ассоциированы с геометрическими лучами
\end{itemize}

Последний пункт является наиболее важным с точки зрения алгоритмов оценки AOA.
Также, из этого свойства канала следует, что количество различимых сильных путей
распространения относительно невелико. Это подтверждено результатами
измерений каналов как для внутренних, так и для наружных сценариев.

Например, результаты измерения AOA в помещении представлены на рис.
\ref{fig:3.1}.  На рис. \ref{fig:3.2} представлены уникальные AOA в случае
уличного сценария <<Манхеттен>>. Можно заметить, что среднее число хорошо
различимых независимых путей распространения $\mu=4.7$, что достаточно
мало.



\begin{figure}[ht]
    \centering
    \includegraphics[width=0.6\linewidth]{figs/fig3.1}
    \caption{ Измерение AOA для определения пути распространения в помещении,
        измеренная мощность показана в полярных координатах, $P$ -- максимум
        измеренной мощности. Геометрические лучи показаны только для позиций
        4.2 и 4.4 \cite{Xu2002}.}
    \label{fig:3.1}
\end{figure}

\begin{figure}[ht]
    \centering
    \includegraphics[width=0.8\linewidth]{figs/fig3.2}
    \caption{
        Зависимость числа различимых путей распространения для различных углов азимута и элевации от расстояния между источником и приемником.
        Среднее значение по всем измерениям составило $\mu=4.7$. Измерения проводились с узкой диаграммой направленности в Манхеттене \cite{Rappaport2015}.}
    \label{fig:3.2}
\end{figure}

На основе рассмотренных работ, можно сделать вывод, что в данной модели
канала чаще всего можно выделить несколько сильнейших путей
распространения и определить их AOA.

<<<<<<< HEAD:sections/1-chapter-one.tex
\subsection{Обзор некоторых методов оценки угловой координаты источника излучения}
=======
\subsection{Обзор методов оценки угловой координаты источника излучения}
\label{sec:review}
>>>>>>> main:sections/1-review.tex

Как показано выше, канал в миллиметровом диапазоне можно представить в виде
набора геометрических лучей.  Самые сильные лучи могут быть
использованы для передачи данных. Как правило, диаграмма направленности
антенны формируется по направлению луча прямой видимости (Line of Sight).
Однако в случае не прямой видимости (Non Line of Sight), может быть
выбран самый сильный отраженный геометрический луч. 

Оценка угла прихода, часто рассматривается в задачах
радиолокации.  Для этих задач  давно разработаны алгоритмы и аппаратные реализации
ещё во времена зарождения радиолокации. Эти алгоритмы совершенствовались, с
появлением фазированных антенных решеток.  
Этот может оказаться очень  полезным с учетом  аппаратных ограничений систем
связи 5G NR -- число цифровых портов обычно мало по сравнению с имеющимся
количеством элементов антенной решетки. 

<<<<<<< HEAD:sections/1-chapter-one.tex
Другой набор алгоритмов пришел из задач спектрального анализа.
В них обычно предполагается, что сигнал каждой антенны принимается независимо.
Эти алгоритмы очень эффективны и дают возможность оценить направления на
несколько целей (лучей) одновременно и имеют сверхразрешающую способность, но c другой стороны, они требуют значительных вычислительных ресурсов.  
=======
Другой набор алгоритмов пришел из задач спектрального анализа.  В них обычно
предполагается, что сигнал каждой антенны принимается независимо.  Эти алгоритмы
очень эффективны и дают возможность оценить направления на несколько целей
(лучей) одновременно и имеют сверх разрешающую способность, но с другой стороны,
они требуют значительных вычислительных ресурсов.  
>>>>>>> main:sections/1-review.tex

В этом разделе мы рассмотрим и систематизируем существующие подходы к оценке
АОА, которые нам удалось найти в открытых литературных источниках.  Будут
представлены их преимущества и недостатки.  На этапе моделирования в следующей
части этой работы, мы сократим этот список и выделим наиболее перспективные
методы.

\subsubsection{Методы Фурье и Бартлетта}
\label{sec:3.2.1}

Простейшая алгоритм оценки AOA в зарубежной литературе называется бимформингом \cite{Tuncer2009, Stoica2005} или методом Фурье \cite{Allen2006}.
Основная идея заключается в максимизации мощности, принятой с определенного
направления.

Обозначим сигнал $\vec{y}(t)$ принятой антенной решеткой от некоторого
удаленного источника
\begin{equation}
    \label{eq:3.1}
    \vec{y}(t) = a(t) \vec{s}(\phi_{src}) + \vec{\xi}(t),
\end{equation}
где ${\vec{s}(\phi_{src})}$ -- фазирующий вектор,
$\phi_{src}$ -- угол прихода (AOA); $\vec \xi$ -- вектор шума.
Каждый элемент фазирующего вектора представляется в виде
\begin{equation}
    \qty{\vec{s}(\phi)}_n = \exp{-i(\vec k (\phi), \vec \rho_n)},
\end{equation}
где $\vec k (\phi_{src})$ -- волновой вектор плоской волны, $\vec\rho_n$ радиус-вектор
$n$-го элемента антенны.
В случае эквидистантной антенной решетки, последнее уравнение приведется
к виду
\begin{equation}
    \qty{\vec s(\phi)}_n = \exp{i2\pi\frac{d}{\lambda}\sin(\phi)n},
\end{equation}
где $d$ -- расстояние между элементами антенной решетки, $\lambda$ -- длина
волны излученного сигнала.

Чтобы получить максимальную мощность с некоторого направления $\phi$ необходимо
сформировать соответствующую диаграмму направленности с помощью
весового вектора антенной решетки $\vec w (\phi) = \vec s(\phi)/\rVert(\vec
    s(\phi))\lVert$.  Тогда, можно найти мощность излученного АР сигнала.
\begin{equation}
    p(\phi) = \abs{\vec w^H (\phi) \vec y}^2.
\end{equation}

Оценкой AOA будет являться значение аргумента $\phi$, обеспечивающего максимум функции $p(\phi)$
\begin{equation}
    \phi^* = \arg\max p(\phi)
\end{equation}

\begin{figure}
    \centering
    \includegraphics[width=\linewidth]{figs/fig3.9}
    \caption{ДН для 16-ти элементной эквидистантной линейной
        решетки ($\frac{d}{\lambda}=0.5$), сформированной в направлении $0^\circ$
        \cite{Tuncer2009}. }
    \label{fig:3.9}
\end{figure}

Для фазированной антенной решетки поиск $\phi^*$ может быть реализован во временной
области с помощью сканирования диаграммой направленности. Если количество
приемников (цифровых портов) равно количеству антенных элементов, искомая
функция $p(\phi)$ может быть оценена в цифровой области \cite{Stoica2005}.  В
литературе этот подход также называется методом Бартлетта \cite{Godara2004}.

Функция $p(\phi)$ примет вид
\begin{equation}
    p(\phi) = \frac{\vec s^H(\varphi) \hat{\vec M} \vec s (\phi)}{N^2},
\end{equation}
где $\hat{\vec M}$ -- оценка корреляционной матрицы принятого сигнала
\begin{equation}
    \hat{\vec M} = \frac{1}{L} \sum\limits_{t=1}^{L} \vec y(t) y^H (t)
\end{equation}

% Уровень боковых лепестков может повлиять на оценку AOA при многолучевом распространении. Чтобы уменьшить уровень боковых
% лепестков антенная решетка взвешивается с некоторой оконной функцией, которая
% устанавливает пространственное распределение амплитуды. 
% Наиболее распространены окна Хэмминга, Ханнинга, Бартлетта, Блэкмана,
% Чебышева и Кайзера. Выбор окна всегда компромисс между
% уровнем боковых лепестков (влиянием помех) и шириной главного лепестка
% (пространственным разрешением).  Например, окно Блэкмана обеспечивает самый низкий уровень
% боковых лепестков и самый широкий главный лепесток.  В отличие от других
% фиксированных окон, окна Кайзера и Чебышева обеспечивают некоторую гибкость в
% настройке результирующих свойств диаграммы направленности. Окно Кайзера
% является аппроксимацией оптимального окна, которое максимизирует относительную
% мощность в главном лепестке \cite{Stoica2005}.  Его часто выбирают вместо
% фиксированного окна, потому что оно имеет более низкий уровень
% боковых лепестков, когда он выбран, чтобы иметь ту же ширину основного
% лепестка, что и соответствующее фиксированное окно (или более узкая ширина
% основного лепестка для данного уровня бокового лепестка).  Окно Чебышева имеет
% то свойство, что пиковый уровень боковых лепестков задается
% постоянным.  При этом ограничении окно обеспечивает минимальную ширину главного
% лепестка. Подробное описание различных оконных функций и их сравнительный анализ можно найти
% в \cite{Stoica2005} \cite{Allen2006}. 

\paragraph{Преимущества}%
\begin{enumerate}
    \item Теоретически, метод Фурье и метод Бартлетта являются оптимальными
          решениями для оценки AOA в случае однолучевого канала.
    \item Легко технически реализуется на конечном устройстве и требует мало
          вычислительных мощностей.
\end{enumerate}

\paragraph{Недостатки}%

\begin{enumerate}
    \item На практике, точность поиска снижается.
          Это происходит, во-первых,  потому что производная функции $p(\phi)$
          в направлении на максимум равна нулю и из-за <<плоской>> вершины сложно
          точно определить точку экстремума. Во-вторых, необходимо
          обеспечить высокую дискретизацию по углу для обеспечения приемлемой
          оценки.
    \item Алгоритм может не подойти в случае быстро движущихся пользователей, если поиск реализован с помощью сканирования во временной области.
    \item Метод обеспечивает разрешающую способность, зависящую
          от ширина главного лепестка ДН. Увеличение отношения сигнал/шум (ОСШ) или времени
          сканирования не приведет к качественному улучшению разрешения.  Это делает
          этот подход малопригодным для оценки многолучевого АОА.
    \item В случае нескольких близко расположенных АОА присутствует
          значительная систематическая ошибка.
\end{enumerate}


\subsubsection{Метод максимального правдоподобия}%
\label{sub:metod_maksimal_nogo_pravdopodobiia}
При наличии нескольких путей распространения, оптимальная оценка AOA может быть
получена с помощью максимально правдоподобной оценки (MLE) \cite{Tuncer2009}. 
Рассмотрим модель сигнала:
\begin{equation}
    \label{eq:3.8}
    \vec y(t) = \sum\limits_{q=1}^{J} a_q(t)\vec s(\phi_q) + \vec \xi(t),
\end{equation}
где $J$ число путей распространения; $a_q(t)$ -- комплексная амплитуда $q$-то
луча, $\vec s(\phi_q)$ -- фазирующий;  $\phi_q$ -- угол прихода (AOA) $q$-то
луча и $\xi(t)$ -- вектор белого гауссового шума.

Для этой модели канала критерий МП может быть записан как критерий минимума
среднеквадратичной ошибки (MMSE)
\begin{equation}
    \label{eq:}
    d(\phi_1, \hdots, \phi_J) = \sum\limits_{t}^{} \abs{\vec(t) -
        \sum\limits_{q=1}^{J} a_q(t) \vec s(\phi_q)}^2 \to \min_{\phi_q}.
\end{equation}
Что можно переписать в виде
\begin{equation}
    \label{eq:}
    d(\phi_1, \hdots, \phi_J) = \sum\limits_{t}^{} \vec y^H(t) \vec
    P_\perp(\phi_1, \hdots, \phi_J) \vec y(t) \to \min_{\phi_q},
\end{equation}
\begin{equation}
    \label{eq:}
    \vec P_\perp(\phi_1, \hdots, \phi_J) = \vec E - \vec S(\vec S^H \vec S)^{-1}
    \vec S^H,
\end{equation}
где $\vec P_\perp$ -- проекционная матрица, 
$\vec S = \qty[\vec s(\phi_1) \dots \vec s (\phi_J)]$. 

Минимизация $d(\phi_1, \hdots, \phi_J)$ в общем случае производится численно и
как правило требует больших вычислительных ресурсов для реализации $J$-мерной
минимизации \cite{Tuncer2009}. 


\paragraph{Преимущества}%
\begin{enumerate}
    \item Обеспечивает оптимальное решение в случае нескольких сильных путей распространения.
\end{enumerate}

\paragraph{Недостатки}%
\begin{enumerate}
<<<<<<< HEAD:sections/1-chapter-one.tex
    \item \hl{Digital antenna array is required}
    \item Требует больших вычислительных затрат
=======
    \item Требует большого количества цифровых портов.
    \item Требует больших вычислительных затрат.
>>>>>>> main:sections/1-review.tex
    \item Не позволяет оценить количество доминирующих путей распространения. Если их количество неизвестно, метод становится неоптимальным. 
\end{enumerate}

\subsubsection{Метод моноимпульса}%
\label{sub:monopulse}

Данная разновидность формирования ДН часто называется моноимпульсом и обычно 
используется в радиолокационных системах для задач слежения. 
Этот алгоритм использует разницу между мощностью двух измеренных лучей как метрику для 
оценки AOA \cite{Tuncer2009}. Вводится следующая функция 
\begin{equation}
    \label{eq:3.11}
    b(\phi) = \frac{1}{\Delta} \qty(\abs{\vec w^H(\phi+0.5 \Delta)\vec y}^2)
    -
    \qty(\abs{\vec w^H(\phi-0.5 \Delta)\vec y}^2) \approx \dv{p(\phi)}{\phi} ,
\end{equation}
где $p(\phi)$ и $\vec w(\phi)$ были определены в разделе \ref{sec:3.2.1}, $\Delta$ -- некоторый скаляр. Тогда, оценка AOA
заключается в поиске такого угла $\phi$, который обеспечивает нуль функции $b(\phi)$
\begin{equation}
    \label{eq:3.12}
    \phi = \arg\qty{b(\phi) = 0}.
\end{equation}

Величина $\Delta$ может быть порядка ширины луча, но  $b(\phi)$ всё равно будет
хорошо аппроксимироваться производной $p(\phi)$, поскольку $b(\phi)$ почти
линейна в большом диапазоне углов около нуля \cite{Tuncer2009}.  

\begin{figure}[h]
    \centering
    \includegraphics[width=\linewidth]{figs/fig3.11}
    \caption{Зависимость метрики $b(\phi)$ для 16-ти элементной ULA \cite{Tuncer2009}. Направление на источник -- 0 град.}
    \label{fig:3.11}
\end{figure}
% Another variant of the algorithm is often called Monopulse Ratio or Amplitude
% Comparison Monopulse \cite{Mosca1969}. This algorithm requires coherent reception with two
% channels (RF-chains): sum and difference. The sum channel is formed with beam
% pattern which has a maximum for a certain direction. The difference beam
% pattern has a null for this direction. In \cite{Kim2018} the algorithm which uses TDM for
% sum and difference channels is proposed and investigated. It employs cycle
% prefix of OFDM signal to receive two identical signals with different beam
% patterns using a single RF-chain and phased antenna array. This approach seems
% promising, but there are some issues related to phase shifter switching delay
% and multipath propagation influence.
% The metric of monopulse ratio is \cite{Kim2018}
% \begin{equation}
%     \label{eq:}
%     \tan(\frac{N}{4}(\phi_{src} - \phi)) =
%     \frac{Im\qty{\sum\limits_{k} y_d(k) y_s^*(k)}}{\sum\limits_{k}
%         \abs{y_s(k)}^2},
% \end{equation}
% where $\phi_{src}$ is actual  AOA;  $\phi$ is roughly estimated AOA via beam
% sweeping (it is the direction of the sum beam); N is number of antenna
% elements; $y_s(k)$ and  $y_d(k)$ are signals f the sum and difference channels
% respectively.
% \begin{equation}
%     \label{eq:}
%     y_s(t) = a(t) \vec w_s^H \vec s(\phi_{src}) + \vec w_s^H \vec \xi(t)
% \end{equation}
% \begin{equation}
%     \label{eq:}
%     y_d(t) = a(t) \vec w_d^H \vec s(\phi_{src}) + \vec w_d^H \vec \xi(t)
% \end{equation}

% For a linear antenna array the corresponding beamforming vectors are
% \begin{equation}
%     \label{eq:}
%     \qty{\vec w_s (\phi)}_n = \exp{i_2\pi \frac{d}{\lambda}} \sin \phi
% \end{equation}
% \begin{equation}
%     \label{eq:}
%     \qty{\vec w_d (\phi)}_{n < \frac n {N}{2}} = - \exp{i_2\pi n \frac{d}{\lambda}} \sin \phi
% \end{equation}
% \begin{equation}
%     \label{eq:}
%     \qty{\vec w_d(\phi)}_{n\geq \frac{N}{2}} = +\exp{i2\pi \frac{d}{\lambda}
%         \sin \phi (n - 0.5N)},
% \end{equation}

% \begin{figure}[htpb]
%     \centering
%     \includegraphics[width=0.6\linewidth]{figs/fig3.12}
%     \caption{Sum and difference beam patterns for monopulse ratio algorithm
%         \cite{Zhu2016}}
%     \label{fig:}
% \end{figure}

% Monopulse ratio is typically used to estimate a single AOA or resolvable angles
% (far spaced targets which are not located within the same beam). However, there
% are some modifications that use a complex monopulse ratio and allow one to
% detect the multiple targets in a certain beam and estimate their angle
% positions \cite{Luoshengbin2016} \cite{Sherman2011}.  


В \cite{Zhu2016} представлена ещё одна реализация метода моноимпульса, не требующая 
поиска нуля функции \eqref{eq:3.11}
\begin{equation}
    \label{eq:3.19}
    \zeta_n = \frac{p(\eta_n - \delta) - p(\eta_n + \delta)}{p(\eta_n - \delta)
        + p(\eta_n + \delta)} =
    \frac{\sin(\psi - \eta_n)\sin\delta}{1 - \cos(\psi - \eta_n)\cos \delta}
\end{equation}

\begin{equation}
    \label{eq:3.20}
    \psi = \eta_n - \arcsin(
    \zeta_n \frac{\sin\delta}{\sin^2 \delta + \zeta^2_n \cos^2\delta}
    -
    \frac{\zeta_n \sqrt{1-\zeta^2_n} \sin \delta \cos \delta}{\sin^2\delta +
        \zeta^2_n \cos^2 \delta}
    )
\end{equation}
где $\psi_n = 2\pi \frac{d}{\lambda} \sin \phi$ -- обобщенный угол 
главного лепестка с азимутальным направлением $\phi$,  $\eta$ -- центральный угол между двумя лучами моноимпульса, 
$n$ -- индекс лучшей пары лучей, $\delta = \frac{\pi}{N}$, $N$ -- число элементов в АР, $p(\eta)$ 
мощность сигнала, измеренной с луча, направленного на обобщенный угол $\eta$.

\begin{figure}[h]
    \centering
    \includegraphics[width=0.45\linewidth]{figs/fig3.13}
    \caption{Сформированные лучи моноимпульса \cite{Zhu2016}}
    \label{fig:}
\end{figure}

Как показано в  \cite{Sherman2011},
этот метод может давать неправильный результат, если фактический AOA
находится вблизи максимума одного из двух лучей, а ОСШ достаточно низкое. 
В \cite{Sherman2011} для решения этой проблемы предлагается использовать ещё два дополнительных луча (см. \ref{fig:3.14}).

\begin{figure}[h]
    \centering
    \includegraphics[width=0.6\linewidth]{figs/fig3.14}
    \caption{Метод моноимпульса с дополнительной парой лучей \cite{Tuncer2009}}
    \label{fig:3.14}
\end{figure}

\paragraph{Преимущества}%
\label{par:preimushchestva}

\begin{enumerate}
    \item На практике, этот метод более точный, чем методы Барлетта или Фурье, поскольку производная достаточно быстро изменяется вблизи AOA.
    \item Может быть реализован на фазированной антенной решетке с одним цифровым портом.
    \item Для оценки необходимо небольшое количество измерений. 
\end{enumerate}

\paragraph{Недостатки}%
\label{par:nedostatki}
\begin{enumerate}
    \item Поскольку является вариацией метода Фурье, его точность будет зависеть 
    от ширины главного лепестка ДН. Увеличение ОСШ или времени оценки не приведет к улучшению результата. 
    \item В случае двух близко расположенных источников сигнала, может наблюдаться значительная систематическая ошибка. 
\end{enumerate}

\subsubsection{Метод Кейпона}%
\label{sub:minimum_variance_distortionless_response_estimator_capon_method_}

Другой алгоритм, основанный на методе Фурье -- Minimum Variance
Distortionless Response Estimator (MVDR), также называемый методом Кейпона \cite{Stoica2005,Allen2006, Godara2004}. 
Основная идея заключается в том, чтобы с помощью формирования ДН 
минимизировать мощность со всех направлений, при постоянном усилении для некоторого направления $\phi$.

\begin{figure}[h]
    \centering
    \includegraphics[width=0.6\linewidth]{figs/fig3.15.png}
    \caption{Основная идея метода Кейпона}
    \label{fig:3.15}
\end{figure}
Выбор весовой вектора в этом случае становится задачей нелинейного программирования 
\cite{Stoica2005, Godara2004}
\begin{equation}
    \label{eq:3.21}
    \vec w(\phi) = \frac{\hat{\vec M}^{-1} \vec s (\phi)}{\vec s^H(\phi)
        \hat{\vec M}^{-1} \vec s(\phi)},
\end{equation}
где $s(\phi)$ -- фазирующий вектор, определенный в \eqref{eq:3.2}, $\vec{\hat M}$ корреляционная матрица \eqref{eq:3.7}.
Искомая функция принимает следующий вид
\begin{equation}
    \label{eq:3.22}
    p(\phi) = \frac{1}{s^H(\phi) \vec M^{-1} \vec s(\phi)}
\end{equation}
Выражение \eqref{eq:3.22} представляет собой принятую мощность. Пики этой функции соответствуют найденными углам прихода. 
\paragraph{Преимущества}%
\begin{enumerate}
    \item Метод Кейпона обеспечивает высокую точность оценки AOA.
    \item Может быть использован для нахождения нескольких AOA, благодаря сверхразрешению. 
    \item Может быть реализован аппаратно, поскольку функция \eqref{eq:3.22} имеет смысл принятой мощности.
\end{enumerate}
\paragraph{Недостатки}%
\begin{enumerate}
    \item Разрешение ограничено, даже если корреляционная матрица $\vec M$
    известна точно. Для улучшения разрешения необходимо увеличивать ОСШ или
    количество элементов АР
    \item Нахождение обратной матрицы, которое необходимо для этого метода, требует больших вычислительных затрат.
\end{enumerate}

