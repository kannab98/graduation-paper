\section{Заключение}

% Описан и продемонстрирован метод для компенсации нелинейных искажений
% усилителя мощности на стороне приемника. В основе метода лежит
% использование обратной амплитудной характеристики усилителя, метод может
% быть применен для нескольких различных типов сигнала и подстроен для
% рассматриваемой задачи. Компенсация была реализована и протестирована на
% симуляторе канального уровня, соответствующего требованиям стандарта 5G
% NR для оценки влияния на производительность системы. Метод применялся для
% существующей модели УМ в диапазоне частот 30-70 ГГц, а также для новой
% модели 100-200 ГГц, основанной на последних исследованиях. В обоих
% случаях, предлагаемый метод продемонстрировал возможность значительно
% улучшить производительность системы в различных условиях (вид сигнала,
% модуляция и др.). Поскольку метод опирается на компенсацию на стороне
% приемника, этот подход может быть эффективен в системе с большим
% количеством простых, дешевых передатчиков с низким энергопотреблением.
% Это позволит устройству передачи снизить общее энергопотребление,
% поскольку в таком случае отсутствует необходимость в предварительной
% обработке сигнала для компенсации нелинейных эффектов усилителя.