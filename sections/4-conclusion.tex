\section{Заключение}

В ходе данной работы был описан и продемонстрирован метод для компенсации
нелинейных искажений усилителя мощности на стороне приемника. В основе
метода лежит использование обратной амплитудной характеристики нелинейного
усилителя, определенной в соответствии с моделью Раппа для твердотельного
усилителя. Разработанный метод может быть модифицирован и применен для
нескольких различных типов сигнала, в частности, алгоритм был подготовлен
для работы с сигналами CP-OFDM и DF-s-OFDM.

Были изучены и проанализированы последние исследования в области создания
усилителей мощности для миллиметрового диапазона. На основе проведенных
исследований была создана обобщенная модель УМ для диапазона частот 100-200
ГГц.

Алгоритм компенсация был реализован и протестирован на симуляторе
канального уровня LLS, соответствующего требованиям стандарта 5G NR для
оценки влияния на общую производительность системы. Метод применялся для
существующей модели УМ в диапазоне частот 30-70 ГГц, а также для новой
модели 100-200 ГГц в широком диапазоне параметров. Для каждой из моделей,
реализованный метод продемонстрировал возможность значительно улучшить
производительность системы в различных условиях (вид сигнала, модуляция и
др.).
\begin{itemize}
    \item Для модели 30-70 ГГц метод компенсации продемонтрировал видимое
    улучшение результата для всех случаев, кроме MCS 22 и SCS 120 кГц. Слабый
    результат компенсации может быть связан с влиянием фазового шума,
    реализованного в LLS.
    \item При использовании модели 100-200 ГГц сигнал часто искажался до
    необратимого состояния, это подчеркивает степень нелинейности
    усилителей в данном диапазоне частот и необходимости предпринятия мер
    при их использовании. Для OFDM сигнала MCS 27 для всех значений SCS при
    добавлении нелинейного УМ BLER достигает единицы и метод компенсации не
    может восстановить информацию.
    \item Для модели 100-200 ГГц метод компенсации продемонтрировал меньшее
    улучшение результата. В частности, для OFDM сигнала MCS 22, производительность
    улучшилась в среднем на 2-3 dB по сравнению со случаем отсутствия
    компенсации. Для DFT-s-OFDM сигнала улучшения наблюдаются в основном для
    MCS 27, производительность в среднем улучшается на 5-10 dB.
\end{itemize}

Поскольку метод опирается на компенсацию на стороне приемника, этот
подход может быть эффективен в системе с большим количеством простых,
дешевых передатчиков с низким энергопотреблением, такой как инфраструктура
IoT. Это позволит устройству передачи снизить общее энергопотребление,
поскольку в таком случае отсутствует необходимость в предварительной
обработке сигнала для компенсации нелинейных эффектов усилителя.