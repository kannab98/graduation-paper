\section{Многолучевые алгоритмы оценки угла прихода сигнала}
\subsection{Иерархический поиск с минимизацией СКО}

Однолучевая версия алгоритма hSearchMMSE, описаная в разделе \ref{sec:4.4.2},
может быть расширина на многолучевую.  Однако, этот алгоритм есть аппроксимация
метода Фурье (непрерывного сканирования лучом), hSearchMMSE имеет характерные
недостатки.  Во-первых, разрешение ограничено шириной луча, но в контексте нашей
системы это не настолько критично. Второй недостаток более серьезный, он связан
с эффектов утечки мощности через боковые лепестки ДН.  Это означает, что мы
можем ошибочно распознать основной путь распространения, обнаруженный боковым
лепестком, как запасной путь.  Чтобы избежать подобной ошибки, необходимо
установить порог мощности для обнаружения запасного пути. Этот порог должен
учитывать утечку мощности через боковые лепестки и шумовое воздействие.

\begin{equation}
    \label{eq:4.57}
    Th_1^{mn} = A_n \frac{\sin^2(0.5 N (\eta_u - \hat \psi_1))}{\sin^2(0.5 (\eta_u - \hat \psi_1))} + 9 \sigma^2,
\end{equation}
\begin{equation}
    \label{eq:4.58}
    Th_2^{mn} = G A_n \frac{\sin^2(0.5 N (\eta_u - \hat \psi_1))}{\sin^2(0.5 (\eta_u - \hat \psi_1))} + 9 \sigma^2,
\end{equation}
где $n$ -- индекс луча базовой станции, $n$ -- индекс луча пользователя, $A_n$
-- <<мощность>> основного луча, включающая в себя ДН базовой станции, $G$ --
ослабление мощности элемента антенной решетки при приеме тыльной стороной
решетки ($-23$ дБ), $\eta_u$ -- направление луча пользователя в обобщенных
координатах, $\hat \psi_1$ -- оцененный угол прихода основного луча, $\sigma^2$
-- мощность шума, множитель $9$ добавлен исходя из правила $3\sigma$. Идея второго слагаемого в выражениях \eqref{eq:4.57},\eqref{eq:4.48} в том, чтобы 
уменьшить вероятность ложной тревоги из-за шума. Порог $Th_1$ используется для следящей решетки, той на которой был определен основой луч, а порог $Th_2$ для запасной решетки.
Величина $A_n$ может быть оценена, используя уравнение 
\begin{equation}
    A_n = \frac{1}{M+1} \sum\limits_{m=-M/2}^{M/2} \hat p_{mn} 
    \frac{\sin^2(0.5(\hat \psi_1 - \chi_m))}{0.5 N_{rx}(\hat \psi_1 - \chi_m)},
\end{equation}
где $\chi_m$ -- обобщенный угол, найденный на этапе сканирования (см. раздел \eqref{sec:4.4.2}), $\hat p_{mn}$ -- 
измеренная мощность на $m$-ом луче UE во время этапа дополнительных измерений и $n$-ом луче BS. Стоит отметить, что $A_n$ для каждого луча оченивается независимо. 

Пошагово алгоритм выглядит следующим образом. 
\begin{enumerate}[label=\textbf{Шаг \arabic*:}]
    \item BS производит сканирование лучом. UE последовательно использует
    каждый луч из кодовой книги \eqref{eq:4.16} для измерения мощности на каждом луче BS.
    Эта процедура выполняется для AIP1 и AIP2. Мощность измерения на этом этапе сохраняется в
    матрицах $\vec P_1$ и $\vec P_2$ соответственно. Каждый элемент матрицы соответствует
    определенным парам лучей UE и BS.
    \item Выбирается лучшая пара лучей UE-BS. Обозначим обобщенный угол лучшего луча как $\eta_{v1}$ и индекс лучшего луча BS как 
    $q_1$. 
    \item Тестируются гипотезы $H_1$, $H_2$, $H_3$ (см. рис. \ref{fig:4.14}) с помощью \eqref{eq:4.34}. Мощность на соседний лучах пользователя 
    ($u=v-1$, $u=v+1$) измеряется на одинаковых лучах BS с индексом $q_1$. Выбирается гипотеза с наибольшей метрикой \eqref{eq:4.34}.
    \item 
\end{enumerate}



