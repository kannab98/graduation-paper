\section{Многолучевые алгоритмы оценки угла прихода сигнала в системе 5G NR}
\subsection{Иерархический поиск с минимизацией СКО}
\label{sec:hSearchMMSE:multipath}

Однолучевая версия алгоритма hSearchMMSE, описаная в разделе \ref{sec:4.4.2},
может быть расширина на многолучевую.  Однако, этот алгоритм есть аппроксимация
метода Фурье (непрерывного сканирования лучом), hSearchMMSE имеет характерные
недостатки.
Во-первых, разрешение ограничено шириной луча, но в контексте нашей
системы это не настолько критично. Второй недостаток более серьезный, он связан
с эффектов утечки мощности через боковые лепестки ДН.  Это означает, что мы
можем ошибочно распознать основной путь распространения, обнаруженный боковым
лепестком, как запасной путь.  Чтобы избежать подобной ошибки, необходимо
установить порог мощности для обнаружения запасного пути. Этот порог должен
учитывать утечку мощности через боковые лепестки и шумовое воздействие.

\begin{equation}
    \label{eq:4.57}
    Th_1^{mn} = A_n
    \frac
    {\sin^2(0.5 N (\eta_u - \hat \psi_1))}
    {\sin^2(0.5 (\eta_u - \hat \psi_1))} + 9 \sigma^2,
\end{equation}
\begin{equation}
    \label{eq:4.58}
    Th_2^{mn} = G A_n \frac{\sin^2(0.5 N (\eta_u - \hat \psi_1))}
    {\sin^2(0.5 (\eta_u - \hat \psi_1))} + 9 \sigma^2,
\end{equation}
где $n$ -- индекс луча базовой станции, $n$ -- индекс луча пользователя, $A_n$
-- <<мощность>> основного луча, включающая в себя ДН базовой станции, $G$ --
ослабление мощности элемента антенной решетки при приеме тыльной стороной
решетки ($-23$ дБ), $\eta_u$ -- направление луча пользователя в обобщенных
координатах, $\hat \psi_1$ -- оцененный угол прихода основного луча, $\sigma^2$
-- мощность шума, множитель $9$ добавлен исходя из правила $3\sigma$.
Идея второго слагаемого в выражениях \eqref{eq:4.57},\eqref{eq:4.48} в том,
чтобы уменьшить вероятность ложной тревоги из-за шума.
Порог $Th_1$ используется
для следящей решетки, той на которой был определен основой луч, а порог $Th_2$
для запасной решетки.  Величина $A_n$ может быть оценена, используя уравнение
\begin{equation}
    A_n = \frac{1}{M+1} \sum\limits_{m=-M/2}^{M/2} \hat p_{mn}
    \frac{\sin^2(0.5(\hat \psi_1 - \chi_m))}{0.5 N_{rx}(\hat \psi_1 - \chi_m)},
\end{equation}
где $\chi_m$ -- обобщенный угол, найденный на этапе сканирования
(см. раздел \eqref{sec:4.4.2}), $\hat p_{mn}$ --
измеренная мощность на $m$-ом луче UE во время этапа дополнительных измерений и
$n$-ом луче BS.
Стоит отметить, что $A_n$ для каждого луча оченивается независимо.

Пошагово алгоритм выглядит следующим образом.
\begin{enumerate}[label=\textbf{Шаг \arabic*:}]
    \item BS производит сканирование лучом. UE последовательно использует
          каждый луч из кодовой книги \eqref{eq:4.16}
          для измерения мощности на каждом луче BS.  Эта процедура
          выполняется для AIP1 и AIP2. Мощность измерения на этом этапе
          сохраняется в матрицах $\vec P_1$ и $\vec P_2$ соответственно.
          Каждый элемент матрицы соответствует
          определенным парам лучей UE и BS.
    \item Выбирается лучшая пара лучей UE-BS. Обозначим обобщенный угол
          лучшего луча как $\eta_{v1}$ и индекс лучшего луча BS как
          $q_1$.
    \item Тестируются гипотезы $H_1$, $H_2$, $H_3$ (см. рис. \ref{fig:4.14}) с
          помощью \eqref{eq:4.34}. Мощность на соседний лучах пользователя
          ($u=v-1$, $u=v+1$) измеряется на одинаковых лучах BS с индексом
          $q_1$. Выбирается гипотеза с наибольшей метрикой \eqref{eq:4.34}.
    \item БС периодически сканирует своими лучами.
          UE последовательно использует каждый луч кодовой книги (4.19) для
          измерить мощность для каждого луча БС. Если выбрана гипотеза $H_2$,
          (4.20) используется для формирования кодовой книги.
          В противном случае используется (4.33).
          Знак <<->> соответствует $H_1$. Знак <<+>> соответствует H3
    \item Мы выполняем алгоритм поиска, представленный в листинге \ref{lst:4.1}, используя
          условие MMSE \eqref{eq:4.31}). 
          Мы используем мощность измеряется для лучшего луча на
          шаге 2 и лучей на шаге 4. Предполагается, что луч БС одинаков.  
          как в лучшей
          паре на шаге 2 (т.е. имеет индекс q1). Пусть $\hat \psi_1$ -- 
          предполагаемая пространственная частота первого путь распространения
    \item Выполняется оценка <<мощности>> используя \eqref{eq:4.59} для 
        каждого луча БС.
    \item Выбираем максимальный элемент матриц $P_1$ или $P_2$ (другая пара лучей
        UE-BS), который превышают пороги \eqref{eq:4.57} или \eqref{eq:4.58}. 
        Пусть это будут индексы
        ($v_2$ и $q_2$). 
        Обратите внимание, что если AIP2 обнаружил основной путь, порог
        $Th_1$ применяется для матрицы $P_2$ (т.е. $P_{2un} > Th_{1un}$) и наоборот. 
        Также мы должны исключить элементы, соответствующие балке, которая была
        выбрана на шаге 2 ($v_1$) для соответствующий АИП. Если выбранный луч 
        $v_2$
        является соседом $v_1$, влияние основного пути на угол атаки оценка
        чрезмерно высока (утечка боковых лепестков). Таким образом, в этом
        случае мы предлагаем изменить AIP и выберите другую лучшую пару лучей.
        Пусть это также будет ($v_2$ и $q_2$).
    \item Мы проверяем гипотезы $H_1$, $H_2$ и $H_3$ для пары лучей, выбранной
          на шаге 7. Гипотеза $H_1$ не проверено, если мы выбрали первый луч UE ($v_2
          = 1$) или луч с индексом ($v_2 = v_1+1$). Гипотеза $H_3$ не тестируется,
          если мы выбрали последний луч UE ($v_2 = 8$) или луч с индексом 
          $(v_2 = v_1-1)$.
    \item БС периодически сканирует своими лучами. UE последовательно использует каждый луч кодовой книги (4.19) для
        измерить мощность для каждого луча БС. Если выбрана гипотеза $H2$,
        \eqref{eq:4.20} используется для формирования кодовой книги.  
        В противном случае используется \eqref{eq:4.33}. 
        Знак <<->> соответствует $H_1$. Знак <<+>> соответствует $H_3$.
    \item Мы выполняем алгоритм поиска, представленный в листинге \ref{lst:4.1}, 
    используя условие MMSE \eqref{eq:4.31}. В уравнение подставляем мощность,
    измеренную для луча, выбранного на шаге 7, и лучей на шаге 9. Луч БС
    предполагается таким же, как и в выбранной паре на шаге 7 (т.е. имеет индекс
    $q_2$). Пусть $\hat \psi_2$ оценивается пространственно частота первого пути
    распространения.
    \item Мы рассчитываем АОА на основе оценок пространственных частот $\hat \psi_1$ и
    $\hat psi_2$. Если выбрано UE- Пара лучей БС связана с АИП1, $\hat \phi_{AOA}=\hat \phi$
    определенной в \eqref{eq:4.21}. 
    Если это связано с AIP2, $\hat \phi_{AOA} = \hat\phi + \pi$. Результат
    получается в радианах.
\end{enumerate}

Временн\'{а}я диаграмма описанноого алгоритма представлена на рис. \ref{fig:4.25}.
Параметры алгоритма представлены в таб. \ref{tab:4.6}.


\begin{table}
    \begin{tabular}{|c|c|c|}
    \end{tabular}
    \caption{}
    \label{tab:4.6}
\end{table}

\subsection[Модифицированный монопульс]{Модифицированный монопульс -- AuxBeam}
\label{sec:AuxBeam:multipath}


Алгоритм вспомогательного луча (см. \ref{sec:AuxBeam:singlepath}) 
может быть расширен в многолучевом
случае аналогично тому, как hSearchMMSE (см. \ref{sec:hSearchMMSE:multipath}). 
На этот алгоритм также влияет проблема боковых лепестков, которая может быть
решается пороговым методом, описанным выше. Таким образом, мы предоставляем
только шаг за шагом описание алгоритма здесь.

\begin{enumerate}[label=\textbf{Шаг \arabic*:}]
    \item 
\end{enumerate}


\subsection{Модифицированный алгоритм бисекций}
\label{sec:Compressive:multipath}
\begin{enumerate}[label=\textbf{Шаг \arabic*:}]
    \item 
    \begin{equation}
        \vec W = 
        \begin{pmatrix}
            \mqty{
                1 & \exp{i\eta_1}\\
                1 & \exp{i\eta_2}\\
                1 & \exp{i\eta_3}\\
                1 & \exp{i\eta_4}\\
            }
            & \mqty{\zmat{4}{6}}
        \end{pmatrix}^T
    \end{equation}
    \item 
    \begin{equation}
        \vec W = 
        \begin{pmatrix}
            \mqty{
                1 & \exp{i\eta_1} & \exp{i2\eta_1} & \exp{i3\eta_1}\\
                1 & \exp{i\eta_2} & \exp{i2\eta_2} & \exp{i3\eta_2}\\
                1 & \exp{i\eta_3} & \exp{i2\eta_3} & \exp{i3\eta_3}\\
                1 & \exp{i\eta_4} & \exp{i2\eta_4} & \exp{i3\eta_4}\\
            }
            & \mqty{\zmat{4}{4}}
        \end{pmatrix}^T
    \end{equation}
    \item 
    \begin{equation}
        \vec W = 
        \begin{pmatrix}
            \mqty{
                1 & \exp{i\eta_1} & \exp{i2\eta_1} & \dots &\exp{i7\eta_1}\\
                1 & \exp{i\eta_2} & \exp{i2\eta_2} & \dots &\exp{i7\eta_2}\\
                1 & \exp{i\eta_3} & \exp{i2\eta_3} & \dots &\exp{i7\eta_3}\\
                1 & \exp{i\eta_4} & \exp{i2\eta_4} & \dots &\exp{i7\eta_4}\\
            }
        \end{pmatrix}^T
    \end{equation}
    \item 
    \begin{equation}
        \vec W = 
        \begin{pmatrix}
            \mqty{
                1 & \exp{i\eta_1} & \exp{i2\eta_1} & \dots &\exp{i7\eta_1}\\
                1 & \exp{i\eta_2} & \exp{i2\eta_2} & \dots &\exp{i7\eta_2}\\
            }
        \end{pmatrix}^T
    \end{equation}
    \item 
    \begin{equation}
        \vec W = 
        \begin{pmatrix}
            \mqty{
                1 & \exp{i\eta_3} & \exp{i2\eta_3} & \dots &\exp{i7\eta_3}\\
                1 & \exp{i\eta_4} & \exp{i2\eta_4} & \dots &\exp{i7\eta_4}\\
            }
        \end{pmatrix}^T
    \end{equation}
\end{enumerate}

