\section{Роль усилителя мощности и влияние нелинейности амплитудной характеристики}
Усилитель мощности (УМ) является ключевым компонентом передатчика,
отвечающим за повышение мощности сигнала, передаваемого устройством или
базовой станцией. УМ также является одним из элементов цепи передатчика с
самым высоким энергопотреблением. Чем больше необходимо усилить сигнал, тем
больше энергии необходимо УМ. Однако при этом, при высокой мощности,
поведение УМ становится нелинейным. Известно, что эффективность УМ,
работающего в радиочастотном диапазоне (RF), может значительно повлиять на
производительность всей передающей системы
в целом \cite{Lie2018}.

В этой главе рассматривается основной принцип работы УМ, влияние
нелинейности на усиливаемый сигнал и математические модели основных
характеристик.


\subsection{Описание принципа работы усилителя мощности}
Одной из основных характеристик УМ является коэффициент усиления (КУ) $G$,
который определяется как отношение выходной мощности $P_{out}$ к входной
$P_{in}$. Часто выражается в дБ:
\begin{equation}
    G_{dB} = 10\log_{10}\left(\frac{P_{out}}{P_{in}}\right)
\end{equation}
КУ УМ зависит от множества факторов, в том числе от свойств элементов,
использованных для создания УМ. Реальный УМ - это нелинейное устройство, КУ
которого не постоянен. Напротив, КУ сильно зависит от свойств
индивидуальных составляющих, входной мощности, частоты сигнала и других
параметров. Однако часто считают, что в некотором диапазоне входных
мощностей и частот КУ является постоянным или достаточно слабо
меняющимся.

Помимо КУ, усилитель мощности часто описывается при помощи двух функций -
амплитудная характеристика (АХ) $F_{AM/AM}$ и фазовая характеристика (ФХ)
$F_{AM/PM}$. Амплитудная характеристика определяет зависимость значения
амплитуды сигнала (напряжения, тока или мощности) на выходе УМ $U_{out}$ от
значения амплитуды сигнала на входе $U_{in}$. Фазовая характеристика
определяет величину сдвига фазы выходного сигнала относительно входного
$\Delta\phi$ в зависимости от амплитуды сигнала на входе $U_{in}$.
\begin{equation}
    U_{out} = F_{AM/AM}(U_{in}), \quad \Delta\phi = F_{AM/PM}(U_{in})
\end{equation}

Для удобства описания и анализа сигнала часто прибегают к использованию
нотации комплексной огибающей $x(t)$. Тогда входной сигнал может быть
записан как 
\begin{equation}
    x(t) = a(t) \exp(i\phi(t)),
\end{equation}
где $a(t)$ - огибающая, $\phi(t)$ - фаза входного сигнала. Отклик УМ $y(t)$
в таком случае будет усиленной и искаженной версией $x(t)$, который может
быть записан как
\begin{equation}
    y(t) = F_{AM/AM}(a(t)) \cdot \exp[i \phi(t) + F_{AM/PM}(a(t))].
\end{equation}

В случае идеального УМ, АХ имеет вид прямой (см. Рис.\ref{fig:1.1}a),
пересекающейся с началом координат. Это означает, что коэффициент усиления
линейного (идеального) усилителя постоянен и не зависит от входного
сигнала. Коэффициент усиления идеального УМ может быть определен как
тангенс угла наклона АХ к оси абсцисс.

\begin{figure}[h!]
    \centering
    \includegraphics[width=0.8\linewidth]{figs/def_pa_char.png}
    \caption{Амплитудные характеристики идеального и реального усилителя мощности}
    \label{fig:1.1}
\end{figure}

Однако как наблюдается на практике, АХ усилителей редко бывают линейными,
ввиду множества факторов, обуславливающих нелинейность этой характеристики
(см. Рис. \ref{fig:1.1}б). При нулевом напряжении на входе, на выходе
усилителя присутствует ненулевое напряжение, обусловленное собственными
шумами усилителя. Из-за этого появляется изгиб в нижней части АХ. При
достаточно больших значениях входной амплитуды, АХ также отклоняется от
прямой. Из-за выхода рабочей точки отдельных элементов усилителя за пределы
рабочего диапазона, возникают нелинейные искажения, в следствие которых
коэффициент усиления сигнала выходит на уровень насыщения.

При этом, АХ реального УМ имеет определенный диапазон входных значений
амплитуд $(U_{in}^{min},U_{in}^{max},)$, при которых искажения практически
отсутствуют и усилитель подобен идеальному. Эта область называется
\textit{динамическим диапазоном усилителя} и выражается как
\begin{equation}
    D = \frac{U_{in}^{max}}{U_{in}^{min}}.
\end{equation}

В связи с ограниченностью динамического диапазона УМ, часто изменяют
входной сигнал таким образом, чтобы итоговая рабочая точка находилась в
нужном диапазоне линейности и усиления. Используется смещение рабочей точки
относительно выходной мощности - OBO (\textit{Англ. - Output back-off}), и
смещение рабочей точки относительно входной мощности - IBO (\textit{Англ. -
Input back-off}). При этом обычно представляется возможным пересчет одной
величины в другую, так как по сути они являются взаимозаменяемыми. Разница
состоит в том, относительно чего происходит сдвиг рабочей точки -
максимальной выходной, либо максимальной входной мощности.

\begin{figure}[h!]
    \centering
    \includegraphics[width=0.7\linewidth]{figs/pa_obo_ibo.png}
    \caption{Смещение рабочей точки усилителя относительно входной и выходной мощности}
    \label{fig:1.2}
\end{figure}

\begin{equation}
    OBO = 10 \cdot \log_{10}\left(\frac{P_{sat}}{P_{out}}\right), \quad
    IBO = 10 \cdot \log_{10}\left(\frac{P_{0}}{P_{in}}\right)
\end{equation}

Большие значения IBO\slash OBO могут обеспечить хорошую линейность АХ,
однако это также приведет к уменьшению средней выходной мощности сигнала.
Таким образом, в реальных применениях определяется наиболее подходящее
значение IBO \slash OBO, обеспечивающее необходимую линейность
характеристики, а также необходимое усиление.


\subsection{Нелинейность и искажение сигнала}
Мощность на выходе УМ увеличивается вместе с ростом входной мощности,
однако, как только уровень выходной мощность достигает определенного
максимума, КУ перестает быть постоянным и УМ входит в область насыщения. В
этой области больше всего проявляется нелинейность УМ - выходная мощность
перестает увеличиваться с ростом входной мощности. Работа УМ в нелинейной
области влечет за собой нелинейные искажения сигнала, которые заключаются в
изменение его формы и фазы.

Рассмотрим в качестве входного сигнала гармонический (синусоидальный)
сигнал (см. рис \ref{fig:pa_distortion_sin}). Если УМ находится в линейном
режиме работы, то на выходе также будет синусоида, отличающаяся только
усилением амплитуды. Однако если усилитель находится в нелинейной области,
на выходе сигнал будет отличаться от входного не только значением
амплитуды, но и формой. Пики синусоиды будут сжиматься нелинейной частью
АХ, что приведет к искажению сигнала на выходе.

\begin{figure}[h!]
    \centering
    \includegraphics[width=0.7\linewidth]{figs/example_pa_signal.png}
    \caption{Преобразование сигнала при прохождении через УМ}
    \label{fig:pa_distortion_sin}
\end{figure}

Сжатие пиков во временной области будет также приводить к искажениям и 
изменениям в частотной
области, а именно к росту ширины спектра. Изначально подаваемый сигнал
имеет очень узкий спектр (дельта-функция на несущей частоте) 

\begin{figure}[h!]
    \centering
    \includegraphics[width=0.9\linewidth]{figs/pa_spectral_growth.png}
    \caption{Влияние сжатия пиков на частотную область сигнала}
    \label{fig:pa_distortion_freq}
\end{figure}




% \subsection{Искажение OFDM сигналов}
% Тут про SC-OFDM, CP-OFDM, бла бла бла
% Или все-таки описать это когда ofdm сигнал опишу?



\subsection{Математическое описание характеристик реальных УМ}

Для моделирования использования УМ в системах мобильной связи часто прибегают к
математическим моделям, описывающим поведение сигнала (усиление, искажение) при 
прохождении через УМ. Исторически модели разделяются на две основных
группы - \textbf{физические} и \textbf{эмпирические} модели \cite{cambridge2008}.

Физические модели требуют знания внутренних электронных компонентов УМ, их
связи, а так же теории, описывающей их взаимодействие. Такие модели подходят
для симуляций на уровне схемы благодаря высокой точности, однако требуют
много вычислительных мощностей и времени, а также детальное описание
структуры и компонентов УМ.

Эмпирические модели используются, когда не известна внутренняя структура УМ,
или когда рассматривается системный уровень моделирования. Эти модели
основаны на результатах измерений и исследований конкретных УМ, на основе
которых были выведены зависимости снятых характеристик УМ (АХ, ФХ) от его
параметров.

Поскольку в данной работе исследуется возможность компенсации нелинейного
искажения на приемнике, то использоваться будет эмпирическая
(поведенческая) модель УМ. Среди таких моделей можно назвать
\textit{Volterra, Saleh, Ghorbani}, а также модели, представляющие собой комбинации
полиномиальных моделей. Все они являются достаточно простыми моделями,
которые отражают нелинейную природу УМ. Простота позволяет оперировать
меньшим количеством параметров усилителя, упрощая обработку в целом. Однако
такие модели не могут быть использованы для описания сложных усилителей,
таки как усилитель \textit{Doherty} \cite{Doherty1936}\cite{3gpp.38.803}.

С другой стороны, в рамках рассматриваемой задачи, а именно компенсации
искажений, внесенных на передатчике из-за нелинейности УМ, использование
более простой модели может быть оправдано. Целью данной работы является
создание метода компенсации нелинейных искажений на приемнике, с основным
мотивом минимизировать обработку на передатчике, а также стоимость
конечного устройства. Рассматриваются именно простые, мало размерные,
дешевые в производстве передатчики, в которых усилитель часто имеет далеко
не лучшие параметры и не отличается высокой эффективностью.

\subsubsection{Модель Раппа}
Для описания искажения амплитуды и фазы при использовании твердотельных УМ
широко используется модель Раппа (\textit{Англ. - Rapp}) \cite{Rapp1991}.
Также существует модифицированная модель Раппа, приведенная в выражении
\ref{eq:Rapp}. Данная модель УМ включена в список моделей УМ в спецификации
3GPP \cite{3gpp.38.803}.

\begin{equation}
    F_{AM/AM}(x) = \frac{G x}{\left( 1 + \abs{\frac{Gx}{V_{sat}}}^{2p}\right)^{1/2p}},
    \quad 
    F_{AM/PM}(x) = \frac{Ax^q}{\left(1+\left(\frac{x}{B}\right)^q\right)},
    \label{eq:Rapp}
\end{equation}
где $F_{AM/AM}, F_{AM/PM}$ - амплитудные и фазовые характеристики
соответственно, $G$ - КУ слабого сигнала, $V_{sat}$ - амплитуда насыщения,
$p$ - показатель гладкости характеристики. Параметры $A,B,q$ - параметры
кривой искажения фазы. В дальнейшем в работе будет использоваться эта
модель для описания влияния УМ на сигнал.

Пример АХ и ФХ для модели Раппа приведены на  рис. \ref{fig:rapp_p_parameters}.
В зависимости от значений параметров $G, V_{sat}, p$ поведение амплитудной
характеристики может сильно варьироваться. Так, при больших значениях $p$
($p\gg 1$), АХ похожа на характеристику идеального УМ, которая ограничена
по максимальной выходной амплитуде (см. рис.\ref{fig:rapp_p_parameters}).
Параметр $V_{sat}$ отвечает за выходную амплитуду насыщения, а $G$ - КУ
слабого сигнала, который показывает, как усилился бы сигнал, если бы УМ был 
идеальным.
\begin{figure}[h!]
    \centering
    \includegraphics[width=0.7\linewidth]{figs/rapp_p.png}
    \caption{Влияние параметра гладкости $p$ на вид амплитудной
    характеристики (Добавить вторую и третью картинку где будут менять G, V)}
    \label{fig:rapp_p_parameters}
\end{figure}

\subsubsection{Параметры модели Раппа для диапазона 30-70 ГГц}
Для вывода параметров базовой модели УМ в диапазоне частот 30-70 ГГц,
компанией Nokia были использованы и исследованы характеристики стандартных
усилителей в соответствующей полосе \cite{nokia163314}. Полученная модель
УМ использована в этой работе для моделирования УМ в диапазоне частот 30-70
ГГц. Амплитудные и частотные характеристики УМ в соответствии с моделью
\cite{nokia163314} приведены на рис. \ref{fig:rapp_nokia}.

Численные значения параметров модели Раппа приведены ниже:
\begin{equation}
    G = 16, \quad V_{sat} = 1.9, \quad p = 1.41
\end{equation}

\begin{figure}[h!]
    \centering
    \includegraphics[width=0.7\linewidth]{figs/rapp_nokia.png}
    \caption{АХ и ФХ усилителя в соответствии с моделью Раппа и
    параметрами для диапазона 30-70 ГГц}
    \label{fig:rapp_nokia}
\end{figure}

\subsection{Характеристики УМ в миллиметровом диапазоне}
Помимо "базовой" модели УМ для диапазона 30-70 ГГц, интерес представляло
моделирование системы в миллиметровом диапазоне, а именно  в диапазоне
частот 100-200 ГГц. Основываясь на работах
\cite{zhang2021}\cite{amadorey2018}\cite{aliyun2020} был сделан вывод о
значительном отличии характеристик УМ в более высоком диапазоне частот.

Характеристики усилителей из исследований приведены на рис.
\ref{fig:pa_research_mean}. Также на рис. \ref{fig:pa_research_mean}
приведены модель для 30-70 ГГц.
\begin{figure}[h]
    \centering
    \includegraphics[width=0.7\linewidth]{figs/pa100mean.png}
    \caption{АХ усилителей на основе данных из
    \cite{zhang2021}\cite{amadorey2018}\cite{aliyun2020}, а также
    полученная усредненная модель для диапазона 100-200 ГГц}
    \label{fig:pa_research_mean}
\end{figure}

Извлеченные кривые амплитудных характеристик были аппроксимированы при
помощи модели Раппа \ref{eq:Rapp}, что позволило собрать параметры УМ для
дальнейшей обработки. Полученные значения, а также частота и технология
для рассматриваемых УМ приведены в Таблице \ref{tab:pa_params}. 

\begin{table}[h]
    \centering
    \begin{tabular}{llcccc}
        \hline
    Источник              & Технология   & Частота, ГГц & $G$  & $V_{sat}$ & $p$ \\ \hline
    \cite{zhang2021}      & 28-нм CMOS   &  135         & 12.26 & 0.9      & 1.93  \\
    \cite{amadorey2018} Рис. 6.2b  & 35-нм mHEMT      &  180  & 10.84 & 0.87 & 0.52 \\
    \cite{amadorey2018} Рис. 6.4b  & 50-нм mHEMT      &  198  & 41.19 & 1.99 & 0.26  \\
    \cite{amadorey2018} Рис. 6.15a & 35-нм GaAs mHEMT &  210  & 7.89  & 0.44 & 0.9  \\
    \cite{aliyun2020} CT       &  130-нм SiGe BiCMOS CT        & 185 & 2.05  & 1.09 & 2.03  \\
    \cite{aliyun2020} CE       & 130-нм SiGe BiCMOS CE         & 185 & 4.08  & 1.41 & 1.91  \\
    \cite{aliyun2020} Рис. 19a & 130-нм SiGe BiCMOS 3-stage CT & 160 & 9.88  & 1.81 & 2.75  \\
    \cite{aliyun2020} Рис. 19b & 130-нм SiGe BiCMOS 3-stage CT & 170 & 14.8  & 1.81 & 1.56  \\
    \cite{aliyun2020} Рис. 19c & 130-нм SiGe BiCMOS 3-stage CT & 185 & 19.29 & 1.86 & 0.87  \\\hline
    \end{tabular}
    \caption{Параметры модели Раппа для УМ в диапазоне частот 100-200 ГГц
    на основе экспериментальных данных и исследований}
    \label{tab:pa_params}
\end{table}

\subsubsection{Новая модель для диапазона частот 100-200 ГГц}
Для исследования применимости нового метода компенсации в диапазоне 100-200
ГГц необходима соответствующая модель УМ, для начальной имплементации ее в
систему с целью внесения соответствующих искажений в сигнал, и последующей
компенсацией внесенных искажений на приемнике. 

Имеющаяся модель \cite{nokia163314} подходит только для диапазона  30-70
ГГц, в нашем случае интерес представляет работа при более высоких частотах.
Модель для 100-200 ГГц была создана путем усреднения параметров $G,
V_{sat}, p$ рассмотренных УМ в соответствующем диапазоне частот (см. Талбицу
\ref{tab:pa_params}). Полученные усредненный параметры для модели Раппа
приведены ниже:
\begin{equation}
    G = 13.59, \quad V_{sat} = 1.35, \quad p = 1.41
\end{equation}


