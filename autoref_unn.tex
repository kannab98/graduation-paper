% \documentclass{unn}
\documentclass{report-statphysicsdep}
\usepackage{gitinfo2}
\usepackage{booktabs} % For \toprule, \midrule and \bottomrule
\usepackage{siunitx} % Formats the units and values
\usepackage{pgfplotstable} % Generates table from .csv
\usepackage{tabularx}

% \addbibresource{library.bib}
\usepackage{enumitem}
\usepackage{subcaption}
\usepackage{soulutf8}

\makeatletter
\patchcmd{\@setref}{\bfseries ??}{\bfseries\hl{??}}{}{}
\makeatother
\pagenumbering{gobble}

\newcommand{\doublesignature}[1][Ivar Nesje Test]{
  \parbox{\textwidth}{
    \centering
    Студент 2 курса магистратуры
    \hfill
    \parbox{5cm}{
      \centering
      \vspace{0.5cm}
      \rule{3cm}{1pt}\\
      Подпись
    }
    % \hfill
    \parbox{5cm}{
      \centering
      \vspace{0.5cm}
      \rule{3cm}{1pt}\\
      Расшифровка
    }
    \hfill
  }
}


\begin{document}





\noindent
{\Large Автореферат ВКР}

\noindent
\textbf{Тема работы}: Метод компенсации нелинейных искажений усилителя
мощности для стандарта мобильной связи 5G NR

\noindent
\textbf{Выполнил}:
% Студент 2 курса магистратура радиофизического
% факультета
Шиков Александр Павлович

% \noindent
\textbf{Введение и актуальность.}
Развитие стандарта мобильной связи 5G NR тесно связано с технологией
Интернета Вещей, а активные исследования по использованию миллиметрового
диапазона позволят увеличить пропускную способность системы. В этом
диапазоне появляются нелинейные искажения, вызванные работой усилителя
мощности (УМ). Несмотря на продвижение в технологии разработки и
проектирования УМ, наблюдаются значительные искажения сигнала при
стандартной мощности передатчика. В случае Интернета Вещей, в
системе множество простых энергоэффективных устройств с низкокачественными
передающими цепями, что связано с малой стоимостью прибора. От этих
параметров зависит выбор метода борьбы с искажениями. Например, доп. обработка
на передатчике может увеличить энергопотребление.

Целями работы являются исследование влияния нелинейности
УМ на различные сигналы, используемые в стандарте 5G,
разработка модели усилителя для диапазона 100-200 ГГц, разработка метода
компенсации нелинейных искажений УМ на приемнике.

% \noindent
\textbf{Содержание работы.}
В работе изучается влияние нелинейности УМ на производительность системы.
Использовалась модель Раппа для диапазона 30-70 ГГц и разработанная в ходе
работы усредненная модель для диапазона 100-200 ГГц. 
Использование нелинейного УМ приводит к значительным искажениям сигналов
CP-OFDM и DFT-s-OFDM. При декодировании возникают ошибки,
понижающие эффективность системы. Для модели 100-200 ГГц искажения имеют
более сильный характер чем для 30-70 ГГц.
Описывается новый метод компенсации нелинейных искажений внесенных УМ на
приемнике для двух типов сигнала. Основа метода заключается в применении
обработки, эквивалентной обработке на передатчике для переноса сигнала во
временную область, где сигнал искажался УМ, а также применении ограниченной
обратной амплитудной характеристики УМ на основе известных параметров для
компенсации искажений.


% \noindent
\textbf{Результаты.}
В работе было изучено влияние искажений нелинейного УМ на эффективность
работы системы. Так же была разработана модель УМ для диапазона 100-200
ГГц. Описан и протестирован метод компенсации искажений на приемнике,
адаптируемый для разных сигналов. Метод
продемонстрировал возможность улучшить производительность системы для числа
случаев (сдвигая кривые BLER на 3-4 дБ), основное преимущество метода
заключается в обработке на приемнике, что минимизирует нагрузку на
передатчик.

\vspace*{\fill}
\doublesignature











\end{document}


